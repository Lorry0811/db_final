% !TEX program = xelatex
\documentclass[a4paper]{article}

% --- 中文設定,使用 xeCJK 套件 ---
\usepackage{xeCJK}
\setCJKmainfont{PingFang TC} % 設定中文字型,蘋方是macOS內建的好選擇
\usepackage{graphicx}
\usepackage{float}

% --- 表格與版面設定(這段就是要加的) ---
\usepackage{booktabs}
\usepackage{array}
% 引入 hyperref 套件
\usepackage{hyperref}

% 設定連結樣式 (建議加上這段,不然預設會有醜醜的紅框)
\hypersetup{
    colorlinks=true,    % 使用顏色來標示連結 (設為 false 則會有方框)
    linkcolor=black,    % 內部連結 (如目錄、章節) 的顏色,設為黑色比較正式
    urlcolor=blue,      % 網址連結的顏色,通常設為藍色讓讀者知道可點擊
    citecolor=black     % 參考文獻引用的顏色
}

\renewcommand{\arraystretch}{1.2}
\setlength{\tabcolsep}{6pt}



% --- 基本文件資訊 ---
\title{114-1 資料庫管理\ 期末專案完整報告}
\author{資管三 B11705061 羅立宸 \and
        資管三 B12705011 黃元翔 \and
        資管三 B12705057 陳以倫}
\date{Dec. 2025} % \today 會自動抓今天的日期



\begin{document}

\maketitle % 產生標題

\begin{center}
    \href{https://github.com/Lorry0811/db_final}{GitHub 專案連結}  % 雙反斜線 \\ 代表換行
    \href{https://www.youtube.com/}{YouTube 影片連結}
\end{center}

\section{系統分析} % 這是第一節的標題
學期初買的教科書超貴,學期末卻堆在角落長灰塵?想買學長姐的二手書,卻不知道去哪裡找?如果你有這些困擾,趕緊上「BookSwap」尋找你需要的二手好物!

「BookSwap」是一個提供給某大學學生刊登及尋找二手教科書與物品的平台 ,主要目的是幫助該校學生解決二手物品(特別是教科書)資訊不流通、交易媒合困難的窘境 。平台上的「刊登」是指一次性的物品出售貼文,每篇貼文都是獨立的,有自己的刊登編號 。交易完成或下架後,該刊登即失效。使用者可以主動舉報可疑貼文或留言,建立社群自我管理的基礎。每次違規都會被記錄,達一定次數(例如 3 次)後,系統會自動停權或封鎖帳號,防止惡意刊登或詐騙帳號重複出現。

根據不同的功能及掌控權限,「BookSwap」系統的用戶可以分為兩種身分,分別是 User 及 Admin 。
User(一般使用者):可依照自身需求,選擇「刊登」想賣的物品,或是「瀏覽/搜尋」平台上由其他人刊登的物品 。若想刊登物品,可透過介面輸入物品的標題、描述、價格、分類,並上傳照片 。如果物品是教科書,還可以關聯到特定「課程」 。如果使用者想購買物品,則可瀏覽平台上現有的刊登,並選擇感興趣的物品進行聯繫。
Admin(管理者):則是「BookSwap」系統的業務經營者,主要負責管理「課程」及「物品分類」的資訊 ,並且可查詢所有使用者的刊登紀錄,收到User舉報後會審核、決定是否移除或警告不適當的刊登內容 。
\subsection{系統功能}
\subsubsection{關於刊登的相關設定}
系統會提供「物品分類」讓使用者選擇,例如:教科書、3C 產品、生活用品等。如果使用者選擇的分類是「教科書」,系統會建議使用者從一個課程列表中選擇想關聯的「課程」 ,例如「資料庫管理」,以利他人搜尋。每則刊登都會有「狀態」,例如:刊登中、預訂中、已售出 。

\subsubsection{給 User 的功能}
在本系統中,User 可以執行以下功能:
\begin{itemize}
    \item 新增刊登:使用者能透過設定物品標題、描述、價格、分類等相關資訊來刊登一項物品 。如果物品是教科書,還可以選擇想關聯的課程 。一旦刊登,系統便會給定一個屬於該刊登的編號 。
    \item 新增留言:使用者能透過在刊登的底下留言,例如"想要"、"有興趣"來表達自己的意願,也能使刊登者知道自己刊登的物品是否有人有意願。
    \item 收藏物品:使用者若看到感興趣的刊登,可將其「加入收藏」,作為一筆新的收藏資料新增至資料庫 。
    \item 管理刊登:使用者如果不想繼續販售,可刪除(或下架)自己刊登的物品 。
    \item 查詢使用者曾經刊登過的物品:使用者可以查詢自身曾建立過的刊登和刊登的所有留言,包括「刊登中」與「已售出」的 。
    \item 查詢使用者收藏的物品:使用者可以查詢自身收藏過的刊登 。
    \item 查詢目前平台上的刊登:使用者可依分類、課程、關鍵字或價格,查詢尚未售出且刊登中的物品 。
    \item 舉報不當刊登:若使用者發現疑似詐騙、違禁品或不當內容的貼文,可透過「舉報」功能提交檢舉。
\end{itemize}

\subsubsection{給 Admin 的功能}
在本系統中,Admin 可以執行以下功能:
\begin{itemize}
    \item 管理課程:業務經營者可對課程資訊進行增刪改查的操作 ,以確保課程列表是最新狀態。
    \item 管理分類:業務經營者可對物品分類進行增刪改查的操作 。
    \item 查詢使用者資訊:業務經營者可查詢所有使用者的活動紀錄,包括該使用者曾經刊登過哪些物品 。
    \item 查詢刊登資訊:業務經營者可查詢所有刊登的詳細資訊,並移除違規(如詐騙或違禁品)的刊登 。
    \item 審核舉報與處理違規:業務經營者會審核舉報內容,若經查屬實,將移除該刊登並記錄違規行為。當使用者違規達一定次數後,系統會自動封鎖該使用者帳號,以維護平台秩序與安全。
\end{itemize}

\section{系統設計}
\subsection{ER Diagram}
Figure 1 是 BookSwap 的 ER Diagram,這個ERD中有三個實體:User, Posting, Comment以及三個關係:Post, Make\_Comments, Have。

其中 User 代表的是使用「BookSwap」的使用者,我們讓使用者用email註冊,並請使用者設定一組username和password,在註冊後我們會給使用者一個獨一無二的ID,並且可以在使用者創建帳戶後,從後台將其變更為管理者,此時該使用者的Admin就會是1,否則為0。

每位使用者都可以將一些他喜歡的post放入自己的favorites清單。另外如果使用者的刊登被下架超過三次,這位使用者也會被封鎖。

Posting 代表的是使用者刊登的物品,每個物品都有一個獨一無二的ID,會有標題、說明文字、價格、狀態(刊登中、已售出、已被檢舉、已下架)、關聯的課程(如果物品是教科書)、課程所屬的類別(Class)比如說法律類、資訊類、文學類。

Comment代表是每個Posting實體下的留言,每則留言都會有獨特的ID、內容、由哪位使用者發出。

Post關係是指一位使用者可以刊登多筆刊登。Make\_Comments關係是指一位使用者可以留多筆言。Have關係是指一筆刊登可以擁有多筆留言。

ERD中可能有部分不符合normalization的設計,這些會在2.2節解決。



\subsection{Relational Database Schema Diagram}
Relational Database Schema Diagram 中總共有7個table,分別是 \texttt{user} (使用者)、\texttt{posting} (貼文)、\texttt{comment} (留言)、\texttt{class} (貼文類別)、\texttt{course} (課程)、\texttt{favorite\_posts} (收藏貼文) 和 \texttt{report} (舉報)。如下圖 Figure 2 所示。


\subsection{Data Dictionary}
資料表共有Relational Database Schema Diagram 所示的七個,各個資料表的欄位相關資訊依序呈現在表 1到表 7。


\subsection{正規化分析}
在 1NF 方面,如果每個關聯的屬性都是 simple 且 single-valued,換句話說,在關
聯中沒有任何一個屬性是 composite 或 multi-valued,則滿足 1NF。

在 2NF 方面,如果關聯中的所有非鍵屬性(non-prime attribute)都完全功能相
依(fully functional dependency)於任一候選鍵(candidate key),也就是沒有出現部
分功能相依性(partial functional dependency),且此關聯滿足 1NF,則滿足 2NF。

在 3NF 方面,如果一個關聯中的非鍵屬性都沒有遞移相依(transitively dependency)於主鍵,則滿足 3NF。因此同樣檢視一下設計的關聯,的確有符合 3NF。在
比 3NF 更嚴謹的 BCNF 方面,要求關聯中的每一個功能相依的箭頭左方都要是超級
鍵(superkey),也就是要確保 X → Y 的 X 一定是超級鍵。我們的 schema 也符合
BCNF。

最後是 4NF,由於「BookSwap」的所有關聯都不存在多值相依(multi-valued dependency),因此滿足 4NF 的條件。


\section{系統實作}
\subsection{資料庫建置方式及資料來源說明}
本系統為確保功能測試之完整性與展示效果,採用自動化腳本批量生成模擬資料(Mock Data)。資料建置並非採用手動輸入,而是透過編寫 PostgreSQL 的 PL/pgSQL 程序化語言腳本(generate/\_fake/\_data.sql),依據預設邏輯自動填充資料庫。具體的資料建置策略與來源說明如下:

\textbf{1. 資料生成策略 (Data Generation Strategy)}

自動化生成:利用 SQL 迴圈(Loop)與隨機函數,動態產生使用者資訊、商品詳情與交易紀錄,模擬系統長期運作下的數據累積。

真實性模擬:
基礎資料:預先匯入真實存在的 20 個科系(如資管、資工、法律等)與 20 門常見課程(如資料庫管理、經濟學原理),確保學術背景的真實感。
內容差異化:針對不同分類(如教科書、3C產品)設定差異化的價格區間與標題格式,並串接外部圖片服務(Placeholder API)模擬商品圖片。

\textbf{2. 邏輯完整性與關聯控制 (Logic \& Integrity Constraints)}

為確保資料符合資料庫正規化原則與業務邏輯,生成過程嚴格遵守以下限制:

參照完整性 (Referential Integrity):嚴格依照「基礎資料 $\rightarrow$ 使用者 $\rightarrow$ 商品刊登 $\rightarrow$ 互動/交易」的順序生成,確保所有外鍵關聯(Foreign Key)正確無誤。

業務邏輯檢查:
買賣邏輯:系統強制限制買家與賣家不得為同一使用者。
狀態連動:僅有狀態為「已售出 (Sold)」的商品才會生成對應訂單;僅有「已完成 (Completed)」的訂單才會產生評價。
價格一致性:訂單成交價格嚴格對應商品刊登時的原始價格。

\textbf{3. 資料規模 (Data Scale)}

目前測試環境共生成約 60,000 筆 資料,涵蓋系統核心的 13 張資料表,規模足以進行壓力測試與查詢效能分析:
使用者:1,000 位(包含雜湊處理過的密碼與隨機餘額)。
商品刊登:12,000 筆(涵蓋上架、保留、售出等不同狀態)。
交易與互動:包含 3,000 筆訂單、5,000 筆交易紀錄及數千筆留言與收藏紀錄。





\subsection{重要功能及對應的 SQL 指令}
\subsection{SQL 指令效能優化與索引建立分析}
\subsection{交易管理}
\subsection{並行控制}

\section{分工資訊}

\section{專案心得}
\end{document}
