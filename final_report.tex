% !TEX program = xelatex
\documentclass[a4paper]{article}

% --- 中文設定,使用 xeCJK 套件 ---
\usepackage{xeCJK}
\setCJKmainfont{PingFang TC} % 設定中文字型,蘋方是macOS內建的好選擇
\usepackage{graphicx}
\usepackage{float}

% --- 表格與版面設定(這段就是要加的) ---
\usepackage{booktabs}
\usepackage{array}
% 引入 hyperref 套件
\usepackage{hyperref}
\usepackage{tabularx}

\usepackage{listings}
\usepackage{xcolor}

\lstset{
    basicstyle=\ttfamily\footnotesize, % 字體
    numbers=left,                      % 行號
    numberstyle=\tiny,
    keywordstyle=\color{blue},
    commentstyle=\color{green!50!black},
    stringstyle=\color{red!60!black},
    breaklines=true,                   % 自動換行
    frame=single                       % 外框
}


% 設定連結樣式 (建議加上這段,不然預設會有醜醜的紅框)
\hypersetup{
    colorlinks=true,    % 使用顏色來標示連結 (設為 false 則會有方框)
    linkcolor=black,    % 內部連結 (如目錄、章節) 的顏色,設為黑色比較正式
    urlcolor=blue,      % 網址連結的顏色,通常設為藍色讓讀者知道可點擊
    citecolor=black     % 參考文獻引用的顏色
}

\renewcommand{\arraystretch}{1.2}
\setlength{\tabcolsep}{6pt}



% --- 基本文件資訊 ---
\title{114-1 資料庫管理\ 期末專案完整報告}
\author{
第十組 \\[6pt]
資管三 B11705061 羅立宸 \\
資管三 B12705011 黃元翔 \\
資管三 B12705057 陳以倫
}
\date{Dec. 2025} % \today 會自動抓今天的日期



\begin{document}

\maketitle % 產生標題

\begin{center}
    \href{https://github.com/Lorry0811/db_final}{GitHub 專案連結}  % 雙反斜線 \\ 代表換行
    \href{https://www.youtube.com/}{YouTube 影片連結}
\end{center}

\section{系統分析} % 這是第一節的標題
學期初買的教科書超貴,學期末卻堆在角落長灰塵?想買學長姐的二手書,卻不知道去哪裡找?如果你有這些困擾,趕緊上「BookSwap」尋找你需要的二手好物!

「BookSwap」是一個提供給某大學學生刊登及尋找二手教科書與物品的平台 ,主要目的是幫助該校學生解決二手物品(特別是教科書)資訊不流通、交易媒合困難的窘境 。平台上的「刊登」是指一次性的物品出售貼文,每篇貼文都是獨立的,有自己的刊登編號 。交易完成或下架後,該刊登即失效。使用者可以主動舉報可疑貼文或留言,建立社群自我管理的基礎。每次違規都會被記錄,達一定次數(例如 3 次)後,系統會自動停權或封鎖帳號,防止惡意刊登或詐騙帳號重複出現。

根據不同的功能及掌控權限,「BookSwap」系統的用戶可以分為兩種身分,分別是 User 及 Admin 。
User(一般使用者):可依照自身需求,選擇「刊登」想賣的物品,或是「瀏覽/搜尋」平台上由其他人刊登的物品 。若想刊登物品,可透過介面輸入物品的標題、描述、價格、分類,並上傳照片 。如果物品是教科書,還可以關聯到特定「課程」 。如果使用者想購買物品,則可瀏覽平台上現有的刊登,並選擇感興趣的物品進行聯繫。
Admin(管理者):則是「BookSwap」系統的業務經營者,主要負責管理「課程」及「物品分類」的資訊 ,並且可查詢所有使用者的刊登紀錄,收到User舉報後會審核、決定是否移除或警告不適當的刊登內容 。
\subsection{系統功能}
\subsubsection{關於刊登的相關設定}
系統會提供「物品分類」讓使用者選擇,例如:教科書、3C 產品、生活用品等。如果使用者選擇的分類是「教科書」,系統會建議使用者從一個課程列表中選擇想關聯的「課程」 ,例如「資料庫管理」,以利他人搜尋。每則刊登都會有「狀態」,例如:刊登中、預訂中、已售出 。

\subsubsection{給 User 的功能}
在本系統中,User 可以執行以下功能:
\begin{itemize}
    \item 新增刊登:使用者能透過設定物品標題、描述、價格、分類等相關資訊來刊登一項物品 。如果物品是教科書,還可以選擇想關聯的課程 。一旦刊登,系統便會給定一個屬於該刊登的編號 。
    \item 新增留言:使用者能透過在刊登的底下留言,例如"想要"、"有興趣"來表達自己的意願,也能使刊登者知道自己刊登的物品是否有人有意願。
    \item 收藏物品:使用者若看到感興趣的刊登,可將其「加入收藏」,作為一筆新的收藏資料新增至資料庫 。
    \item 管理刊登:使用者如果不想繼續販售,可刪除(或下架)自己刊登的物品 。
    \item 查詢使用者曾經刊登過的物品:使用者可以查詢自身曾建立過的刊登和刊登的所有留言,包括「刊登中」與「已售出」的 。
    \item 查詢使用者收藏的物品:使用者可以查詢自身收藏過的刊登 。
    \item 查詢目前平台上的刊登:使用者可依分類、課程、關鍵字或價格,查詢尚未售出且刊登中的物品 。
    \item 舉報不當刊登:若使用者發現疑似詐騙、違禁品或不當內容的貼文,可透過「舉報」功能提交檢舉。
\end{itemize}

\subsubsection{給 Admin 的功能}
在本系統中,Admin 可以執行以下功能:
\begin{itemize}
    \item 管理課程:業務經營者可對課程資訊進行增刪改查的操作 ,以確保課程列表是最新狀態。
    \item 管理分類:業務經營者可對物品分類進行增刪改查的操作 。
    \item 查詢使用者資訊:業務經營者可查詢所有使用者的活動紀錄,包括該使用者曾經刊登過哪些物品 。
    \item 查詢刊登資訊:業務經營者可查詢所有刊登的詳細資訊,並移除違規(如詐騙或違禁品)的刊登 。
    \item 審核舉報與處理違規:業務經營者會審核舉報內容,若經查屬實,將移除該刊登並記錄違規行為。當使用者違規達一定次數後,系統會自動封鎖該使用者帳號,以維護平台秩序與安全。
\end{itemize}

\section{系統設計}
\subsection{ER Diagram}
Figure 1 是 BookSwap 的 ER Diagram,這個ERD中有三個實體:User, Posting, Comment以及三個關係:Post, Make\_Comments, Have。

其中 User 代表的是使用「BookSwap」的使用者,我們讓使用者用email註冊,並請使用者設定一組username和password,在註冊後我們會給使用者一個獨一無二的ID,並且可以在使用者創建帳戶後,從後台將其變更為管理者,此時該使用者的Admin就會是1,否則為0。

每位使用者都可以將一些他喜歡的post放入自己的favorites清單。另外如果使用者的刊登被下架超過三次,這位使用者也會被封鎖。

Posting 代表的是使用者刊登的物品,每個物品都有一個獨一無二的ID,會有標題、說明文字、價格、狀態(刊登中、已售出、已被檢舉、已下架)、關聯的課程(如果物品是教科書)、課程所屬的類別(Class)比如說法律類、資訊類、文學類。

Comment代表是每個Posting實體下的留言,每則留言都會有獨特的ID、內容、由哪位使用者發出。

Post關係是指一位使用者可以刊登多筆刊登。Make\_Comments關係是指一位使用者可以留多筆言。Have關係是指一筆刊登可以擁有多筆留言。

ERD中可能有部分不符合normalization的設計,這些會在2.2節解決。



\subsection{Relational Database Schema Diagram}
Relational Database Schema Diagram 中總共有7個table,分別是 \texttt{user} (使用者)、\texttt{posting} (貼文)、\texttt{comment} (留言)、\texttt{class} (貼文類別)、\texttt{course} (課程)、\texttt{favorite\_posts} (收藏貼文) 和 \texttt{report} (舉報)。如下圖 Figure 2 所示。


\subsection{Data Dictionary}
資料表共有Relational Database Schema Diagram 所示的七個,各個資料表的欄位相關資訊依序呈現在表 1到表 11。

\begin{table}[H]
    \centering
    \renewcommand{\arraystretch}{1.1}     % 行距稍微放大一點點就好
    \setlength{\tabcolsep}{3pt}           % 欄與欄之間縮小
    \footnotesize                         % 字小一號
    \begin{tabular}{p{2.1cm} p{2.6cm} p{2.4cm} p{0.9cm} p{3.1cm} p{3.6cm}}
    \hline
    Column Name & Meaning & Data Type & Key & Constraint & Domain \\
    \hline
    u\_id          & 使用者編號       & INT UNSIGNED   & PK & Not Null, Auto\_Increment & 正整數流水號 \\
    is\_admin      & 是否為管理者     & TINYINT(1)     &    & Not Null, Default 0       & \{0:一般使用者,1:管理者\} \\
    user\_name     & 使用者名稱       & VARCHAR(50)    &    & Not Null, Unique          &  \\
    email          & 電子郵件         & VARCHAR(100)   &    & Not Null, Unique          & 符合 Email 格式 \\
    password       & 密碼             & VARCHAR(8)     &    & Not Null                  & 登入用密碼 \\
    status         & 使用者帳號狀態   & ENUM(...)      &    & Not Null, Default 'normal'& \{'normal','banned'\} \\
    violation\_cnt & 違規次數         & INT UNSIGNED   &    & Not Null, Default 0       & 達一定次數自動停權 \\
    balance        & 錢包餘額         & INT            &    & Not Null, Default 0       & 單位:元,新台幣 \\
    \hline
    \end{tabular}
    \caption{資料表 USER 的欄位資訊}
\end{table}

\begin{table}[H]
    \centering
    \renewcommand{\arraystretch}{1.1}     % 行距
    \setlength{\tabcolsep}{3pt}           % 欄距
    \footnotesize                         % 字體稍微縮小
    \begin{tabular}{p{2.1cm} p{1.5cm} p{3.5cm} p{0.9cm} p{3.1cm} p{3.6cm}}
    \hline
    Column Name & Meaning & Data Type & Key & Constraint & Domain \\
    \hline
    p\_id       & 刊登編號           & INT UNSIGNED        & PK  & Not Null, Auto\_Increment                     & 每筆刊登的流水號 \\
    description & 描述               & TEXT                &     & Not Null                                      & 刊登內容說明文字 \\
    price       & 價格               & DECIMAL(10,2)       &     & Not Null, CHECK (price $\ge$ 0)              & 新台幣金額 \\
    status      & 刊登狀態           & ENUM('listed', 'reserved', 'sold', 'removed', 'reported') & & Not Null, Default 'listed' & listed:刊登中;reserved:預訂中;sold:已售出;removed:下架;reported:被舉報 \\
    u\_id       & 刊登者使用者編號   & INT UNSIGNED        & FK  & Not Null, REFERENCES user(u\_id) \\ 
                &                    &                     &     & ON DELETE CASCADE ON UPDATE CASCADE          & 對應發文者之 user.u\_id \\
    course\_id  & 關聯課程編號       & INT UNSIGNED        & FK  & Nullable, REFERENCES course(course\_id) \\ 
                &                    &                     &     & ON DELETE SET NULL ON UPDATE CASCADE         & 若為教科書則指向對應課程,非教科書可為 NULL \\
    \hline
    \end{tabular}
    \caption{資料表 POSTING 的欄位資訊}
\end{table}
    
\begin{table}[H]
    \centering
    \renewcommand{\arraystretch}{1.1}
    \setlength{\tabcolsep}{3pt}
    \footnotesize
    \begin{tabular}{p{2.1cm} p{2.6cm} p{2.4cm} p{0.9cm} p{3.1cm} p{3.6cm}}
    \hline
    Column Name & Meaning & Data Type & Key & Constraint & Domain \\
    \hline
    c\_id    & 留言編號           & INT UNSIGNED & PK & Not Null, Auto\_Increment & 每則留言的流水號 \\
    content  & 留言內容           & TEXT         &    & Not Null                   & 留言文字內容 \\
    u\_id    & 留言者使用者編號   & INT UNSIGNED & FK & Not Null, FK $\rightarrow$ "user"(u\_id), ON DELETE CASCADE ON UPDATE CASCADE & 對應留言者之 user.u\_id \\
    p\_id    & 所屬刊登編號       & INT UNSIGNED & FK & Not Null, FK $\rightarrow$ posting(p\_id), ON DELETE CASCADE ON UPDATE CASCADE & 對應貼文之 posting.p\_id \\
    \hline
    \end{tabular}
    \caption{資料表 COMMENT 的欄位資訊}
\end{table}
    
\begin{table}[H]
    \centering
    \renewcommand{\arraystretch}{1.1}
    \setlength{\tabcolsep}{3pt}
    \footnotesize
    \begin{tabular}{p{2.1cm} p{2.6cm} p{2.4cm} p{0.9cm} p{2.4cm} p{3.6cm}}
    \hline
    Column Name & Meaning & Data Type & Key & Constraint & Domain \\
    \hline
    class\_id   & 類別編號         & INT UNSIGNED   & PK & Not Null, Auto\_Increment & 貼文類別流水號 \\
    class\_name & 類別名稱         & VARCHAR(100)   &    & Not Null, Unique & 如「資訊類」、「法律類」、「文學類」等 \\
    \hline
    \end{tabular}
    \caption{資料表 CLASS 的欄位資訊}
\end{table}

\begin{table}[H]
    \centering
    \renewcommand{\arraystretch}{1.1}
    \setlength{\tabcolsep}{3pt}
    \footnotesize
    \begin{tabular}{p{2.1cm} p{2.6cm} p{2.4cm} p{0.9cm} p{3.1cm} p{3.6cm}}
    \hline
    Column Name & Meaning & Data Type & Key & Constraint & Domain \\
    \hline
    course\_id   & 課程編號         & INT UNSIGNED & PK & Not Null, Auto\_Increment & 每門課程的流水號 \\
    course\_name & 課程名稱         & VARCHAR(120) &    & Not Null                  & 如「資料庫管理」、「線性代數」等課程名稱 \\
    class\_id    & 所屬類別編號     & INT UNSIGNED & FK & Not Null, FK $\rightarrow$ class(class\_id), ON DELETE RESTRICT ON UPDATE CASCADE & 對應之貼文類別編號 class.class\_id \\
    \hline
    \end{tabular}
    \caption{資料表 COURSE 的欄位資訊}
\end{table}

\begin{table}[H]
    \centering
    \renewcommand{\arraystretch}{1.1}
    \setlength{\tabcolsep}{3pt}
    \footnotesize
    \begin{tabular}{p{2.1cm} p{2.6cm} p{2.4cm} p{0.9cm} p{3.1cm} p{3.6cm}}
    \hline
    Column Name & Meaning & Data Type & Key & Constraint & Domain \\
    \hline
    u\_id      & 使用者編號         & INT        & PK, FK & Not Null, FK $\rightarrow$ "user"(u\_id), ON DELETE CASCADE ON UPDATE CASCADE & 收藏者之 user.u\_id \\
    p\_id      & 刊登編號           & INT        & PK, FK & Not Null, FK $\rightarrow$ posting(p\_id), ON DELETE CASCADE ON UPDATE CASCADE & 被收藏貼文之 posting.p\_id \\
    added\_time& 加入收藏時間       & TIMESTAMP  &        & Default CURRENT\_TIMESTAMP & 系統記錄加入最愛的時間點 \\
    \hline
    \end{tabular}
    \caption{資料表 FAVORITE\_POSTS 的欄位資訊}
\end{table}

\begin{table}[H]
    \centering
    \renewcommand{\arraystretch}{1.1}
    \setlength{\tabcolsep}{3pt}
    \footnotesize
    \begin{tabular}{p{2.1cm} p{2.6cm} p{2.4cm} p{0.9cm} p{3.1cm} p{3.6cm}}
    \hline
    Column Name & Meaning & Data Type & Key & Constraint & Domain \\
    \hline
    u\_id       & 檢舉者使用者編號 & INT UNSIGNED & PK, FK & Not Null, FK $\rightarrow$ "user"(u\_id), ON DELETE CASCADE ON UPDATE CASCADE & 發出檢舉的使用者編號 \\
    p\_id       & 被檢舉刊登編號   & INT UNSIGNED & PK, FK & Not Null, FK $\rightarrow$ posting(p\_id), ON DELETE CASCADE ON UPDATE CASCADE & 被檢舉貼文之編號 \\
    content     & 檢舉內容         & TEXT         &        & Not Null                                      & 檢舉原因、說明文字 \\
    report\_time& 檢舉時間         & TIMESTAMP    & PK     & Not Null, Default CURRENT\_TIMESTAMP          & 系統記錄檢舉送出時間 \\
    \hline
    \end{tabular}
    \caption{資料表 REPORT 的欄位資訊}
\end{table}
    
\begin{table}[H]
    \centering
    \renewcommand{\arraystretch}{1.1}
    \setlength{\tabcolsep}{3pt}
    \footnotesize
    \begin{tabular}{p{2.1cm} p{2.6cm} p{2.4cm} p{0.9cm} p{3.1cm} p{3.6cm}}
    \hline
    Column Name & Meaning & Data Type & Key & Constraint & Domain \\
    \hline
    order\_id   & 訂單編號         & SERIAL       & PK  & Not Null, Auto\_Increment                     & 每筆訂單的唯一流水號 \\
    buyer\_id   & 買家使用者編號   & INT          & FK  & Not Null, FK $\rightarrow$ "user"(u\_id)      & 下單者之 user.u\_id \\
    p\_id       & 刊登編號         & INT          & FK  & Not Null, FK $\rightarrow$ posting(p\_id)     & 對應的商品刊登編號 \\
    deal\_price & 成交價格         & INT          &     & Not Null                                      & 成交當下之新台幣金額 \\
    order\_date & 下單時間         & TIMESTAMP    &     & Default CURRENT\_TIMESTAMP                    & 系統自動記錄建立時間 \\
    status      & 訂單狀態         & VARCHAR(20)  &     & Default 'completed'                           & \{'completed','cancelled'\} 等狀態值 \\
    \hline
    \end{tabular}
    \caption{資料表 ORDERS 的欄位資訊}
\end{table}
    
\begin{table}[H]
    \centering
    \renewcommand{\arraystretch}{1.1}
    \setlength{\tabcolsep}{3pt}
    \footnotesize
    \begin{tabular}{p{2.1cm} p{2.6cm} p{2.4cm} p{0.9cm} p{3.1cm} p{3.6cm}}
    \hline
    Column Name & Meaning & Data Type & Key & Constraint & Domain \\
    \hline
    record\_id  & 交易紀錄編號     & SERIAL       & PK  & Not Null, Auto\_Increment                     & 每筆金流紀錄的唯一流水號 \\
    u\_id       & 帳戶使用者編號   & INT          & FK  & Not Null, FK $\rightarrow$ "user"(u\_id)      & 這筆交易所屬之 user.u\_id \\
    amount      & 金額             & INT          &     & Not Null                                      & 正數:收入/儲值;負數:支出 \\
    trans\_type & 交易類型         & VARCHAR(20)  &     & & 例如 'top\_up','payment','income' 等 \\
    trans\_time & 交易時間         & TIMESTAMP    &     & Default CURRENT\_TIMESTAMP                    & 系統自動記錄之時間戳記 \\
    \hline
    \end{tabular}
    \caption{資料表 TRANSACTION\_RECORD 的欄位資訊}
\end{table}
    
\begin{table}[H]
    \centering
    \renewcommand{\arraystretch}{1.1}
    \setlength{\tabcolsep}{3pt}
    \footnotesize
    \begin{tabular}{p{2.1cm} p{2.6cm} p{2.4cm} p{0.9cm} p{3.1cm} p{3.6cm}}
    \hline
    Column Name & Meaning & Data Type & Key & Constraint & Domain \\
    \hline
    review\_id   & 評價編號           & SERIAL      & PK  & Not Null, Auto\_Increment              & 每筆評價的流水號 \\
    order\_id    & 對應訂單編號       & INT         & FK  & Not Null, FK $\rightarrow$ orders(order\_id)   & 被評價之訂單 \\
    reviewer\_id & 評價者使用者編號   & INT         & FK  & Not Null, FK $\rightarrow$ "user"(u\_id)       & 發出評價者 \\
    target\_id   & 被評價者使用者編號 & INT         & FK  & Not Null, FK $\rightarrow$ "user"(u\_id)       & 評價的對象 \\
    rating       & 評分               & INT         &     & Not Null, CHECK (rating BETWEEN 1 AND 5)       & 1\textasciitilde5 分整數 \\
    comment      & 評價內容           & TEXT        &     &                                            & 文字評論,可為空 \\
    created\_at  & 建立時間           & TIMESTAMP   &     & Default CURRENT\_TIMESTAMP                & 評價建立之時間 \\
    \hline
    \end{tabular}
    \caption{資料表 REVIEW 的欄位資訊}
\end{table}

\begin{table}[H]
    \centering
    \renewcommand{\arraystretch}{1.1}
    \setlength{\tabcolsep}{3pt}
    \footnotesize
    \begin{tabular}{p{2.1cm} p{2.6cm} p{2.4cm} p{0.9cm} p{3.1cm} p{3.6cm}}
    \hline
    Column Name & Meaning & Data Type & Key & Constraint & Domain \\
    \hline
    msg\_id      & 訊息編號           & SERIAL     & PK  & Not Null, Auto\_Increment           & 每則私訊的流水號 \\
    sender\_id   & 寄件者使用者編號   & INT        & FK  & Not Null, FK $\rightarrow$ "user"(u\_id) & 發送訊息者 \\
    receiver\_id & 收件者使用者編號   & INT        & FK  & Not Null, FK $\rightarrow$ "user"(u\_id) & 接收訊息者 \\
    content      & 訊息內容           & TEXT       &     & Not Null                             & 私訊文字內容 \\
    sent\_time   & 傳送時間           & TIMESTAMP  &     & Default CURRENT\_TIMESTAMP           & 訊息送出時間 \\
    is\_read     & 是否已讀           & BOOLEAN    &     & Not Null, Default FALSE              & TRUE:已讀;FALSE:未讀 \\
    \hline
    \end{tabular}
    \caption{資料表 MESSAGE 的欄位資訊}
\end{table}
    
    
\subsection{正規化分析}
在 1NF 方面,如果每個關聯的屬性都是 simple 且 single-valued,換句話說,在關
聯中沒有任何一個屬性是 composite 或 multi-valued,則滿足 1NF。

在 2NF 方面,如果關聯中的所有非鍵屬性(non-prime attribute)都完全功能相
依(fully functional dependency)於任一候選鍵(candidate key),也就是沒有出現部
分功能相依性(partial functional dependency),且此關聯滿足 1NF,則滿足 2NF。

在 3NF 方面,如果一個關聯中的非鍵屬性都沒有遞移相依(transitively dependency)於主鍵,則滿足 3NF。因此同樣檢視一下設計的關聯,的確有符合 3NF。在
比 3NF 更嚴謹的 BCNF 方面,要求關聯中的每一個功能相依的箭頭左方都要是超級
鍵(superkey),也就是要確保 X → Y 的 X 一定是超級鍵。我們的 schema 也符合
BCNF。

最後是 4NF,由於「BookSwap」的所有關聯都不存在多值相依(multi-valued dependency),因此滿足 4NF 的條件。


\section{系統實作}
\subsection{資料庫建置方式及資料來源說明}
本系統為確保功能測試之完整性與展示效果,採用自動化腳本批量生成模擬資料(Mock Data)。資料建置並非採用手動輸入,而是透過編寫 PostgreSQL 的 PL/pgSQL 程序化語言腳本(generate/\_fake/\_data.sql),依據預設邏輯自動填充資料庫。具體的資料建置策略與來源說明如下:

\textbf{1. 資料生成策略 (Data Generation Strategy)}

自動化生成:利用 SQL 迴圈(Loop)與隨機函數,動態產生使用者資訊、商品詳情與交易紀錄,模擬系統長期運作下的數據累積。

真實性模擬:
基礎資料:預先匯入真實存在的 20 個科系(如資管、資工、法律等)與 20 門常見課程(如資料庫管理、經濟學原理),確保學術背景的真實感。
內容差異化:針對不同分類(如教科書、3C產品)設定差異化的價格區間與標題格式,並串接外部圖片服務(Placeholder API)模擬商品圖片。

\textbf{2. 邏輯完整性與關聯控制 (Logic \& Integrity Constraints)}

為確保資料符合資料庫正規化原則與業務邏輯,生成過程嚴格遵守以下限制:

參照完整性 (Referential Integrity):嚴格依照「基礎資料 $\rightarrow$ 使用者 $\rightarrow$ 商品刊登 $\rightarrow$ 互動/交易」的順序生成,確保所有外鍵關聯(Foreign Key)正確無誤。

業務邏輯檢查:
買賣邏輯:系統強制限制買家與賣家不得為同一使用者。
狀態連動:僅有狀態為「已售出 (Sold)」的商品才會生成對應訂單;僅有「已完成 (Completed)」的訂單才會產生評價。
價格一致性:訂單成交價格嚴格對應商品刊登時的原始價格。

\textbf{3. 資料規模 (Data Scale)}

目前測試環境共生成約 60,000 筆 資料,涵蓋系統核心的 13 張資料表,規模足以進行壓力測試與查詢效能分析:
使用者:1,000 位(包含雜湊處理過的密碼與隨機餘額)。
商品刊登:12,000 筆(涵蓋上架、保留、售出等不同狀態)。
交易與互動:包含 3,000 筆訂單、5,000 筆交易紀錄及數千筆留言與收藏紀錄。





\subsection{重要功能及對應的 SQL 指令}

在第 1.1 節中,我們介紹了 BookSwap 平台提供給一般使用者(User)與管理員(Admin)
的主要功能,以及系統在背景中自動執行的一些維護操作。本小節將以具體情境為例,
說明這些功能在實作上對應到的 SQL 指令或資料庫函數呼叫。第 3.2.1 小節聚焦於
給一般使用者使用的功能,第 3.2.2 小節則說明管理員相關的查詢與維護操作,
最後第 3.2.3 小節則整理由系統本身運行的「系統級指令」。

\subsubsection{給 User 的功能}

\begin{enumerate}
    \item \textbf{刊登二手書:新增一筆商品刊登}\\
    若要實現「刊登二手書」功能,假設情境為:
    「使用者代號 \texttt{u\_id} 為 10 的學生,想刊登一本標題為
    『資料庫管理(第 3 版)』的教科書,售價 \texttt{price} 為 450 元,
    並指定分類 \texttt{class\_id} 為 2、對應課程 \texttt{course\_id} 為 15。」\\
    對應的 SQL 指令如下。系統會在 \texttt{posting} 資料表新增一筆刊登紀錄,
    初始狀態 \texttt{status} 設為 \texttt{'listed'} 代表可供購買。

\begin{verbatim}
INSERT INTO posting (u_id, title, description, price, status,
                     class_id, course_id)
VALUES (10,
        '資料庫管理(第 3 版)',
        '九成新,書內有少量筆記,可面交',
        450,
        'listed',
        2,
        15)
RETURNING p_id;
\end{verbatim}

    透過 PostgreSQL 的 \texttt{RETURNING} 子句,系統可以立即取得新建立的
    \texttt{p\_id},後續若需要再為該刊登新增多張圖片,便能將此編號插入
    \texttt{posting\_images} 資料表中。

\begin{verbatim}
INSERT INTO posting_images (p_id, image_url, display_order)
VALUES (/* 上一步取得的 p_id */, 
        'https://example.com/images/book1.jpg',
        1);
\end{verbatim}

    \item \textbf{搜尋可購買的書籍:依課程名稱與價格篩選刊登}\\
    使用者在前端可以輸入關鍵字、選擇課程或分類,來查詢目前可購買的刊登。
    假設情境為:「使用者想查詢課程名稱內含『資料庫』,並且價格介於
    0 到 500 元之間,且商品狀態為可購買(\texttt{listed})的刊登。」\\
    對應的 SQL 指令如下。系統會回傳符合條件的刊登資訊,包含課程名稱、
    分類名稱與賣家帳號:

\begin{verbatim}
SELECT p.p_id,
       p.title,
       p.price,
       p.status,
       u.username      AS seller_username,
       co.course_name,
       cl.class_name
FROM posting AS p
JOIN "user" AS u   ON p.u_id = u.u_id
LEFT JOIN course AS co ON p.course_id = co.course_id
LEFT JOIN class  AS cl ON p.class_id = cl.class_id
WHERE p.status = 'listed'
  AND co.course_name ILIKE '%資料庫%'
  AND p.price BETWEEN 0 AND 500
ORDER BY p.created_at DESC;
\end{verbatim}

    此查詢會利用我們在 \texttt{posting(status)}、\texttt{posting(price)}、
    \texttt{course(course\_id)} 等欄位上建立的索引,降低掃描資料表的成本。

    若使用者改以關鍵字搜尋書名/描述,我們則會使用 PostgreSQL 全文搜尋,
    對應到資料庫中以 \texttt{GIN} 索引加速的 \texttt{to\_tsvector} 欄位:

\begin{verbatim}
SELECT p.p_id, p.title, p.price, p.status
FROM posting AS p
WHERE p.status = 'listed'
  AND to_tsvector('english', p.title || ' ' || COALESCE(p.description, ''))
      @@ plainto_tsquery('english', 'database');
\end{verbatim}

    \item \textbf{將刊登加入收藏清單:新增收藏紀錄}\\
    若要實現「加入收藏」功能,假設情境為:
    「使用者代號 \texttt{u\_id} 為 10,想把刊登編號 \texttt{p\_id} 為 123 的書籍加入收藏。」\\
    對應的 SQL 指令如下。系統會在 \texttt{favorite\_posts} 資料表中新增一筆紀錄,
    若日後再次查詢收藏清單,即可快速找到這筆刊登。

\begin{verbatim}
INSERT INTO favorite_posts (u_id, p_id)
VALUES (10, 123);
\end{verbatim}

    \item \textbf{購買書籍:呼叫交易函數 \texttt{purchase\_book}}\\
    當使用者決定購買某本書時,前端會將買家代號與刊登編號送往後端,
    由後端透過 PostgreSQL 的 PL/pgSQL 函數 \texttt{purchase\_book} 執行完整的交易流程。
    假設情境為:「使用者 \texttt{u\_id = 10} 想購買刊登 \texttt{p\_id = 123} 的書籍。」\\
    對應的 SQL 呼叫如下:

\begin{verbatim}
SELECT purchase_book(10, 123);
\end{verbatim}

    此函數內部會依序檢查刊登是否存在、狀態是否為 \texttt{'listed'}、
    買家餘額是否足夠、是否為自己購買自己的商品,並在同一個交易中
    同步完成扣款、入帳、更新刊登狀態與建立訂單,確保購買流程具有 ACID 特性。

    \item \textbf{在刊登底下留言:新增公開留言}\\
    為了讓買家可以在刊登底下詢問細節(例如是否可面交、是否有畫線),
    我們提供「公開留言」功能。假設情境為:
    「使用者 \texttt{u\_id = 10} 想在刊登 \texttt{p\_id = 123} 底下留言『請問可以在總圖面交嗎?』。」\\
    對應的 SQL 指令如下:

\begin{verbatim}
INSERT INTO comment (p_id, u_id, content)
VALUES (123, 10, '請問可以在總圖面交嗎?');
\end{verbatim}

    系統會自動記錄留言時間 \texttt{created\_at},後續在刊登頁面讀取留言時,
    會依照時間排序顯示。

    \item \textbf{傳送私訊:建立一則一對一訊息}\\
    若買家與賣家希望進一步溝通(例如交換 Line、約定面交時間),
    系統會透過 \texttt{message} 資料表記錄雙方的一對一訊息。
    假設情境為:「使用者 \texttt{u\_id = 10} 想私訊刊登的賣家 \texttt{u\_id = 25}。」\\
    對應的 SQL 指令如下:

\begin{verbatim}
INSERT INTO message (sender_id, receiver_id, content)
VALUES (10, 25, '您好,請問這本書還在嗎?可以約下週一面交嗎?');
\end{verbatim}

    當訊息送出後,前端會根據 \texttt{is\_read} 欄位顯示未讀提示,
    並透過索引 \texttt{idx\_message\_receiver\_id} 與 \texttt{idx\_message\_is\_read}
    加速收件匣查詢。
\end{enumerate}

\subsubsection{給 Admin 的功能}

\begin{enumerate}
    \item \textbf{管理課程與分類(增刪改查)}\\
    管理員可以維護系上課程、系所與分類資訊,讓刊登可以正確對應到實際課程。
    假設情境為:「管理員想新增一門課程『資料庫管理』,
    屬於『資訊管理學系』且歸類在『必修』這個分類中。」\\
    對應的 SQL 指令如下。首先新增系所與分類(若尚未存在),再新增課程:

\begin{verbatim}
INSERT INTO department (dept_name)
VALUES ('資訊管理學系');

INSERT INTO class (class_name, description)
VALUES ('必修', '系上必修課程');

INSERT INTO course (course_code, course_name, dept_id, class_id)
VALUES ('IM3001', '資料庫管理', 1, 1);
\end{verbatim}

    若日後需要修改課程代碼或名稱,管理員可以使用 \texttt{UPDATE} 指令:

\begin{verbatim}
UPDATE course
SET course_name = '資料庫管理(含實作)'
WHERE course_id = 1;
\end{verbatim}

    \item \textbf{處理舉報:審核留言/刊登/逃單舉報}\\
    當使用者發現不當內容或交易糾紛時,可以透過 \texttt{report} 資料表送出舉報。
    管理員在後台審核時,會將狀態從 \texttt{'pending'} 更新為
    \texttt{'approved'} 或 \texttt{'rejected'}。假設情境為:
    「管理員 \texttt{u\_id = 1} 審核編號為 50 的舉報,決定通過此舉報。」\\
    對應的 SQL 指令如下:

\begin{verbatim}
UPDATE report
SET status      = 'approved',
    reviewed_by = 1,
    reviewed_at = CURRENT_TIMESTAMP
WHERE report_id = 50;
\end{verbatim}

    當舉報被標記為 \texttt{'approved'} 時,觸發器 \texttt{update\_violation\_count}
    會根據舉報類型(刊登、留言或逃單)自動找到對應的目標使用者,
    並將其 \texttt{violation\_count} 加一;當違規次數累積到三次,
    另一個觸發器 \texttt{auto\_block\_user} 會自動將該帳號的
    \texttt{is\_blocked} 欄位設為 \texttt{TRUE},達到半自動的風紀管理效果。

    \item \textbf{查詢使用者整體表現與風險狀態}\\
    為了讓管理員能快速掌握每位使用者在平台上的活動概況,
    我們在資料庫中建立視圖 \texttt{v\_user\_statistics},整理了
    刊登數量、售出數量、購買金額、總收入、平均評分與收藏數等指標。
    若管理員想查詢 \texttt{u\_id = 10} 的統計資訊,對應的 SQL 指令如下:

\begin{verbatim}
SELECT *
FROM v_user_statistics
WHERE u_id = 10;
\end{verbatim}

    此視圖將 \texttt{"user"}、\texttt{posting}、\texttt{orders}、
    \texttt{review} 與 \texttt{favorite\_posts} 等多個資料表的資訊整合在一起,
    管理員只需一個查詢即可檢視使用者是否為高風險對象(例如違規次數偏高、
    評價過低或逃單紀錄較多)。

    \item \textbf{查詢熱門書籍與課程統計}\\
    為了協助平台調整推薦策略與營運方向,我們另外建立了
    \texttt{v\_popular\_books}、\texttt{v\_course\_statistics}、
    \texttt{v\_class\_statistics} 等視圖。若管理員想查詢目前收藏數較多的熱門書籍,
    對應的 SQL 指令如下:

\begin{verbatim}
SELECT *
FROM v_popular_books
WHERE favorite_count >= 5
ORDER BY favorite_count DESC, comment_count DESC;
\end{verbatim}

    若要觀察特定課程(例如 \texttt{IM3001} 資料庫管理)的整體交易狀況,
    則可使用下列查詢:

\begin{verbatim}
SELECT *
FROM v_course_statistics
WHERE course_code = 'IM3001';
\end{verbatim}

    藉由這些彙總視圖,管理員可以了解哪些課程或分類上的二手書需求特別高,
    作為未來功能優化與行銷活動的依據。
\end{enumerate}

\subsubsection{系統級指令}

除了由使用者或管理員直接觸發的操作外,BookSwap 也在資料庫端設計了多個
「系統級」的 SQL 函數與觸發器,用來在背景中維護金流紀錄與帳號風紀,
避免出現資料不一致的情況。

\begin{enumerate}
    \item \textbf{計算使用者平均評分:\texttt{calculate\_user\_rating}}\\
    為了在使用者頁面顯示賣家的整體評價,我們在資料庫中實作了
    \texttt{calculate\_user\_rating(p\_user\_id INT)} 函數,用來計算指定使用者
    作為被評價者(\texttt{target\_id})時,其所有評分的平均值。
    假設系統需要取得 \texttt{u\_id = 10} 的平均評分,對應的 SQL 呼叫如下:

\begin{verbatim}
SELECT calculate_user_rating(10);
\end{verbatim}

    此函數會忽略尚未有評分的情況(回傳 0),並將結果四捨五入至小數點後兩位,
    方便直接顯示在前端介面上。

    \item \textbf{取得使用者銷售統計:\texttt{get\_user\_sales\_stats}}\\
    另一個系統級函數 \texttt{get\_user\_sales\_stats(p\_user\_id INT)} 則會回傳一個 JSON,
    其中包含使用者售出的書籍數量、總收入、目前上架中的刊登數量,以及平均評分等資訊。
    假設系統想取得 \texttt{u\_id = 10} 的銷售統計,對應的 SQL 呼叫如下:

\begin{verbatim}
SELECT get_user_sales_stats(10);
\end{verbatim}

    這個函數主要用於後台儀表板與使用者個人頁面,讓前端可以一次取得多個統計欄位,
    減少與資料庫往返的次數。

    \item \textbf{訂單完成時自動更新刊登狀態與金流紀錄}\\
    在第 3.4 節中我們已詳細說明購買流程的交易管理。這裡補充說明與之對應的
    系統級指令:當 \texttt{orders} 資料表中新增或更新一筆訂單,
    且其狀態為 \texttt{'completed'} 時,觸發器
    \texttt{update\_posting\_status\_on\_order} 與
    \texttt{record\_transaction\_on\_order} 會自動執行下列 SQL 片段:

\begin{verbatim}
-- 1. 將對應的刊登狀態改為 sold
UPDATE posting
SET status    = 'sold',
    updated_at = CURRENT_TIMESTAMP
WHERE p_id = NEW.p_id AND status = 'listed';

-- 2. 在 transaction_record 中新增兩筆金流紀錄
INSERT INTO transaction_record (u_id, amount, trans_type)
VALUES (NEW.buyer_id, -NEW.deal_price, 'payment');  -- 買家付款

INSERT INTO transaction_record (u_id, amount, trans_type)
VALUES (seller_id, NEW.deal_price, 'income');       -- 賣家收入
\end{verbatim}

    如此一來,即使應用程式端只需插入或更新一筆訂單紀錄,資料庫也能自動確保
    刊登狀態與金流紀錄保持一致,避免出現「訂單完成但金流未記錄」或
    「金流有紀錄但刊登仍顯示可購買」等不一致情況。
\end{enumerate}

\subsection{SQL 指令效能優化與索引建立分析}
\subsubsection{User 表索引}
我們觀察到系統中最頻繁的操作之一是與使用者相關的查詢,例如登入驗證、判斷帳號是否為管理員、篩選遭封鎖的用戶等。由於 email 與 username 是辨識使用者身分的重要欄位,且在註冊、登入與權限驗證流程中會被大量使用,因此若每次查詢都需在資料表逐筆比對,勢必造成效能負擔。

此外,is\_admin 與 is\_blocked 這兩個欄位經常被用於篩選條件,例如後台管理需要快速定位管理員帳號,或系統需要查詢遭封禁的使用者狀態。若無索引輔助,資料庫必須進行全表掃描 (Full Table Scan),在使用者數量成長後,將會造成明顯延遲。

因此,為提升查詢效率,我們在 user 表中針對 email、username、is\_admin 與 is\_blocked 建立索引,以加快系統在使用者查詢行為中的回應速度。其語法如下。

\begin{lstlisting}[language=SQL]
CREATE INDEX IF NOT EXISTS idx_user_email
ON "user"(email);

CREATE INDEX IF NOT EXISTS idx_user_username
ON "user"(username);

CREATE INDEX IF NOT EXISTS idx_user_is_admin
ON "user"(is_admin);

CREATE INDEX IF NOT EXISTS idx_user_is_blocked
ON "user"(is_blocked);
\end{lstlisting}
上述索引的建立,使常見查詢如「搜尋 E-mail 是否已註冊」、「驗證使用者名稱是否合法」、「取得所有管理員帳號」與「查詢封鎖用戶」能夠避免全表掃描,提升資料檢索效率,並強化系統的使用體驗與可擴展性。

\subsubsection{Posting 表索引}
在系統的運作流程中,posting 資料表扮演核心角色,包含貼文內容、分類、價格、課程代碼與貼文建立時間等資訊。由於平台上的貼文瀏覽、篩選與搜尋操作頻繁,若每次查詢皆需逐筆比對,將造成系統效能下降。因此,我們決定針對常用查詢欄位建立索引,以提高查詢速度。

其中,u\_id 會用於取得使用者的所有貼文;status 會頻繁用於顯示有效貼文或過期貼文的篩選;class\_id 與 course\_id 則有助於快速查詢特定課程或分類下的貼文內容。price 與 created\_at 亦為排序及篩選熱門或最新貼文時不可或缺的條件,因此透過索引,我們可以有效降低查詢時間。

除了以上欄位索引外,我們亦為 title 與 description 建立全文搜尋 (Full Text Search) 索引,使用 PostgreSQL GIN + Tsvector 技術,以提升關鍵字搜尋效率與回傳精準度,使使用者能夠快速找到符合需求的貼文內容。

其語法如下。

\begin{lstlisting}[language=SQL]
-- Posting 表索引
CREATE INDEX IF NOT EXISTS idx_posting_u_id ON posting(u_id);
CREATE INDEX IF NOT EXISTS idx_posting_status ON posting(status);
CREATE INDEX IF NOT EXISTS idx_posting_class_id ON posting(class_id);
CREATE INDEX IF NOT EXISTS idx_posting_course_id ON posting(course_id);
CREATE INDEX IF NOT EXISTS idx_posting_price ON posting(price);
CREATE INDEX IF NOT EXISTS idx_posting_created_at ON posting(created_at);

-- 全文搜尋索引(PostgreSQL)
CREATE INDEX IF NOT EXISTS idx_posting_title_search 
ON posting USING gin(to_tsvector('english', title));

CREATE INDEX IF NOT EXISTS idx_posting_description_search 
ON posting USING gin(to_tsvector('english', description));

\end{lstlisting}

透過此索引設計,系統能夠在貼文數量增加時仍維持快速回應,特別是篩選搜尋、課程查詢、價格排序與全文關鍵字查詢皆大幅提升效能,有助於貼文瀏覽與交易流程更加順暢。

\subsubsection{Comment 表索引}
在互動功能中,留言系統是使用者參與貼文內容的重要環節,comment 資料表則用於儲存各篇貼文的留言紀錄。為了確保系統能夠即時顯示留言、查詢特定貼文相關的討論內容,我們針對常用的查詢條件進行索引設計。

其中,p\_id (貼文編號) 是最常用來查詢留言的條件,系統在顯示貼文下方留言時會頻繁依據此欄位提取資料,因此建立索引能有效降低 Full Table Scan 的成本。同時,u\_id 可加速查詢某位使用者留下的所有留言,例如顯示個人留言記錄或追蹤違規帳號留言行為。至於 created\_at 則常作為排序依據(最新留言在前),亦有助於以時間維度讀取留言串的效能。

其語法如下。
\begin{lstlisting}[language=SQL]
CREATE INDEX IF NOT EXISTS idx_comment_p_id ON comment(p_id);
CREATE INDEX IF NOT EXISTS idx_comment_u_id ON comment(u_id);
CREATE INDEX IF NOT EXISTS idx_comment_created_at ON comment(created_at);

\end{lstlisting}

透過此索引設置,系統在讀取留言串、顯示討論內容、回朔使用者留言紀錄時皆能更快速回應,並確保在留言數增加的情況下仍具備高擴充性與良好使用體驗。

\subsubsection{Report 表索引}
在檢舉與審核流程中,report 資料表負責儲存使用者針對貼文所提報的違規紀錄,因此查詢效率直接影響到審核速度與管理端的使用體驗。由於平台在運作過程中,管理者會大量針對檢舉紀錄進行查詢、篩選與排序,我們在設計上對常用查詢欄位建立索引,以強化系統在多筆檢舉情況下的效能。

其中,reporter\_id 主要用於追蹤同一位使用者提出的所有檢舉行為,可協助判斷是否存在濫用申訴或重複檢舉。p\_id 則對應貼文本身,可加速取得某則貼文涉及的所有檢舉事件,特別是熱門貼文被大量舉報時更能有效減少查詢負擔。status 為審查狀態,建立索引後可快速篩選未處理、通過或駁回的案件,大幅提升後台審核流程效率。而 created\_at 則讓系統能依時間排序與取得最新檢舉紀錄,對時序分析與審核排序十分關鍵。

\begin{lstlisting}[language=SQL]
CREATE INDEX IF NOT EXISTS idx_report_reporter_id ON report(reporter_id);
CREATE INDEX IF NOT EXISTS idx_report_p_id ON report(p_id);
CREATE INDEX IF NOT EXISTS idx_report_status ON report(status);
CREATE INDEX IF NOT EXISTS idx_report_created_at ON report(created_at);
\end{lstlisting}

透過上述索引設計,後台管理者能更快速定位檢舉來源、查詢涉及問題的貼文、掌握處理進度並依時間排序案件,使審核流程具備更高反應速度與可處理上限,在資料量增加時亦保持良好延展性。

\subsubsection{Orders 表索引}
在交易流程設計中,orders 資料表負責紀錄使用者之間的交易訂單資訊,包含購買者、對應貼文、訂單狀態與下單時間等欄位。由於訂單查詢是整體系統運作的重要核心,例如買家查看訂單紀錄、賣家確認商品是否售出、後台篩選訂單狀態等場景都會大量依賴此資料表,因此若無索引輔助,在訂單數量增加後將出現顯著查詢延遲。

其中,buyer\_id 用於取得某位使用者的購買紀錄,使平台能快速顯示歷史訂單、評價來源或交易分析結果。p\_id 則對應 posting,可快速確認某篇貼文所產生的所有訂單,有助於追蹤熱門貼文的交易量。status 常用於後台審核與訂單處理,如顯示已完成、待付款或已取消的訂單狀態,索引能使篩選條件查詢更流暢。最後,order\_date 使系統能快速依時間排序訂單,在報表生成、營運分析與近期交易查詢中具有重要作用。

索引建立語法如下。

\begin{lstlisting}[language=SQL]
CREATE INDEX IF NOT EXISTS idx_orders_buyer_id ON orders(buyer_id);
CREATE INDEX IF NOT EXISTS idx_orders_p_id ON orders(p_id);
CREATE INDEX IF NOT EXISTS idx_orders_status ON orders(status);
CREATE INDEX IF NOT EXISTS idx_orders_order_date ON orders(order_date);
\end{lstlisting}
透過以上索引配置,系統能在訂單量成長後仍保持查詢效率,包含查詢購買紀錄、對應貼文之訂單、依狀態分類訂單處理、以及依時間排序交易資料,都能顯著降低查詢延遲,確保交易流程與後台統計分析運作順暢。

\subsubsection{Transaction Record 表索引}
在平台金流與點數制度中,transaction\_record 表用於儲存使用者的錢包交易紀錄,包含轉入、扣款、退款、購買等不同類型的金流行為。由於此資料與訂單、貼文交易、錢包餘額顯示等功能密切相關,交易紀錄的快速查詢對整體使用體驗與後台審計都具有關鍵影響。因此,我們針對常用查詢條件建立索引,以確保金流紀錄在長期使用下仍能維持高效運作。

其中,u\_id 是最常用的查詢欄位,用於取得某位使用者的所有錢包紀錄,例如顯示充值紀錄、消費歷史或違規退款事件;建立索引後,可顯著加速此類查詢需求。trans\_type 則用於類型識別,方便後台統計特定交易類型的發生次數,如統計每日扣款量、每日充值量等,索引可提升分類查詢的效率。trans\_time 則非常適合用於時間排序與報表分析,當需取得近期交易或回朔歷史紀錄時,有索引的情況下可避免進行全表掃描,使查詢更具延展性。

建立索引之 SQL 如下所示:
\begin{lstlisting}[language=SQL]
CREATE INDEX IF NOT EXISTS idx_transaction_u_id ON transaction_record(u_id);
CREATE INDEX IF NOT EXISTS idx_transaction_trans_type ON transaction_record(trans_type);
CREATE INDEX IF NOT EXISTS idx_transaction_trans_time ON transaction_record(trans_time);
\end{lstlisting}
透過以上索引設計,交易紀錄在查詢使用者歷史、類型分類統計、依時間排序時皆具備更佳查詢效率,並能在資料量累積下保持後台分析、錢包查詢與系統審計的即時性與穩定度。

\subsubsection{Review 表索引}
評價系統是平台信任機制的重要組成,review 資料表負責儲存訂單完成後的評分與評論內容,用於反映交易品質、使用者行為、以及是否具備良好風評。由於評價資料經常用於排序、查詢、身份追溯、後台稽核等功能,因此若缺乏索引,當評論量逐漸成長後查詢將明顯變慢。因此,我們針對高頻查詢欄位建立索引以提升系統效能。

其中,order\_id 可快速定位某筆交易是否已有評價,以及查看買賣雙方對單一訂單的回饋;reviewer\_id 則用於查看某位使用者發表的所有評價,有助於分析買家或賣家在平台上的使用紀錄;target\_id 則代表評論的對象,即買方/賣方/貼文持有者,用於評估其整體評價紀錄與可信度;rating 則在排序、篩選高低分數評論時提供性能優勢,使平台得以快速呈現優質交易或找出風險帳號。

SQL 建置語法如下:
\begin{lstlisting}[language=SQL]
CREATE INDEX IF NOT EXISTS idx_review_order_id ON review(order_id);
CREATE INDEX IF NOT EXISTS idx_review_reviewer_id ON review(reviewer_id);
CREATE INDEX IF NOT EXISTS idx_review_target_id ON review(target_id);
CREATE INDEX IF NOT EXISTS idx_review_rating ON review(rating);
\end{lstlisting}
透過以上索引設計,系統能快速取得訂單的評價記錄、查詢使用者過往評價行為、查看某一帳號的信用評分並依照分數排序評論,不論是買家瀏覽評價或後台稽查帳號,都能擁有更佳查詢體驗與處理效率,並使平台信任機制得以有效運作。

\subsubsection{Message 表索引}
私訊系統是平台中使用者互動的主要方式之一,message 資料表負責存放雙方溝通內容,包括訊息傳送者、接收者、傳送時間與已讀狀態等欄位。由於訊息查詢行為可能頻繁且即時性需求高,例如顯示聊天室歷史訊息、讀取未讀訊息提醒或依時間排序對話內容,因此建立索引能有效提升整體訊息讀取與傳遞效率。

其中,sender\_id 與 receiver\_id 用於快速查詢特定使用者之間的訊息往來,當聊天室或對話視窗開啟時,系統可透過索引直接定位相關訊息而避免全表掃描。在通知系統中,is\_read 作為未讀訊息篩選條件,索引可使平台快速判斷某位使用者是否有未查看訊息。sent\_time 則常用於排序訊息流,如呈現最新對話、顯示時間紀錄、撈取歷史訊息等情境,索引能確保即使訊息量增加仍具備即時讀取能力。

建立索引的 SQL 語法如下:
\begin{lstlisting}[language=SQL]
CREATE INDEX IF NOT EXISTS idx_message_sender_id ON message(sender_id);
CREATE INDEX IF NOT EXISTS idx_message_receiver_id ON message(receiver_id);
CREATE INDEX IF NOT EXISTS idx_message_is_read ON message(is_read);
CREATE INDEX IF NOT EXISTS idx_message_sent_time ON message(sent_time);
\end{lstlisting}
透過此索引配置,平台能更迅速完成私訊紀錄讀取、未讀訊息提醒、依時間排序訊息流等任務,並確保聊天系統在大量訊息累積時依然能保持順暢,不會因資料量成長而使訊息開啟延遲或查詢速度下降。

\subsubsection{Favorite Posts 表索引}
收藏機制讓使用者能快速儲存、追蹤與再次瀏覽感興趣的貼文,而 favorite\_posts 資料表即為此功能的核心儲存來源。平台在顯示會員的收藏清單、檢查某篇貼文是否已加入最愛、或分析熱門收藏貼文時,都會頻繁存取此表。因此我們針對常見查詢條件建立索引,以提升系統存取性能。

其中,u\_id 用於查詢某位使用者的收藏紀錄。在前端點擊「查看我的收藏」或刷新收藏頁面時,索引能避免逐筆搜索,顯著減少查詢時間。p\_id 則可協助後台追蹤某篇貼文的收藏熱度,常用於統計、熱門排序或推薦系統。另一欄位 added\_time 則能使系統依收藏時間排序,如顯示最新收藏、歷史追蹤或時間分群分析,索引可避免大量排序運算,進一步降低反應延遲。

建立索引的 SQL 如下:
\begin{lstlisting}[language=SQL]
CREATE INDEX IF NOT EXISTS idx_favorite_u_id ON favorite_posts(u_id);
CREATE INDEX IF NOT EXISTS idx_favorite_p_id ON favorite_posts(p_id);
CREATE INDEX IF NOT EXISTS idx_favorite_added_time ON favorite_posts(added_time);
\end{lstlisting}
透過上述索引設計,系統在顯示收藏列表、查詢貼文是否已加入收藏、以及依時間排序收藏紀錄時皆能更快速、低延遲地回應。此外,索引能在資料量累積後維持效能,使收藏功能具備良好擴展性並保持使用體驗流暢。


\subsubsection{Posting Images 表索引}
貼文圖片為使用者認知商品資訊的重要來源,posting\_images 資料表負責儲存貼文對應圖片的檔案位置與顯示順序。由於前端在呈現貼文內容時,會頻繁依照貼文編號載入多張圖片,並依照排序進行顯示,因此建立索引可有效提升貼文圖片讀取速度並降低查詢延遲。

其中,p\_id 是最核心的查詢欄位,用於顯示某篇貼文所有圖片。例如使用者點開貼文或切換頁面時,系統會大量依據此欄位查詢,因此建立索引能避免大量 Full Table Scan。另一欄位 display\_order 則負責控制顯示排列順序,如縮圖排序、主圖優先顯示等,索引能加速依序排列查找圖片順序的流程,確保商品圖片載入過程更流暢。

SQL 建立索引語法如下:
\begin{lstlisting}[language=SQL]
CREATE INDEX IF NOT EXISTS idx_posting_images_p_id ON posting_images(p_id);
CREATE INDEX IF NOT EXISTS idx_posting_images_display_order ON posting_images(display_order);
\end{lstlisting}
透過以上索引設定,系統能更快速載入貼文圖片、依照順序呈現圖檔內容,並在貼文圖片量增加後仍維持查詢與排序效率,提升整體瀏覽體驗與資料表的可擴展性。

\subsubsection{Course 表索引}
課程資料是整個平台分類貼文、搜尋條件與使用者瀏覽邏輯中的關鍵基礎,course 資料表負責儲存科系代碼、分類層級、課程代碼等資訊,並與 posting、favorite\_posts、order 等功能相互關聯。因此,我們針對常用的查詢欄位建立索引,使課程搜尋與過濾過程能在大量資料狀況下依然維持快速回應。

其中,dept\_id 可用於篩選特定系所課程,在搜尋某系課程或分類貼文時能明顯降低查詢延遲。class\_id 則對應課程分類層級,使平台能依班別(Class Category)快速定位課程,常用於分類頁與課程列表展開時。course\_code 為課程的唯一識別欄位,使用者在精確查詢或比對課程資料時,索引能避免進行全表掃描,並加速後台比對、資訊串接與貼文標註流程。

SQL 建立索引語法如下:
\begin{lstlisting}[language=SQL]
CREATE INDEX IF NOT EXISTS idx_course_dept_id ON course(dept_id);
CREATE INDEX IF NOT EXISTS idx_course_class_id ON course(class_id);
CREATE INDEX IF NOT EXISTS idx_course_code ON course(course_code);
\end{lstlisting}
透過以上索引機制,課程分類、代碼查詢與系所篩選流程皆能顯著降低讀取成本,並確保課程資料在被頻繁引用至貼文、收藏、訂單與搜尋模組時仍可快速存取,使整體系統的課程導向查詢更加流暢、可擴展且具備長期效能優勢。



\subsubsection{索引效果}

\subsection{交易管理}
在本系統中,購買流程被視為一個必須同時滿足原子性(Atomicity)、一致性(Consistency)、隔離性(Isolation)與持久性(Durability)的複合操作。因此,我們選擇將交易邏輯下放到資料庫端,以 PostgreSQL 的 PL/pgSQL 函數實作完整的交易流程,而非單純在應用程式層中依序執行多個獨立的 SQL 指令。在 \texttt{005\_add\_functions.sql} 中,我們定義了一個 \texttt{purchase\_book(p\_buyer\_id INT, p\_posting\_id INT)} 函數,負責處理從檢查商品狀態、驗證餘額、扣款與入帳,到更新刊登狀態與建立訂單的全部步驟。當使用者在前端按下購買按鈕時,系統會呼叫此函數,由資料庫在單一交易(transaction)中完成所有相關更新。

在交易過程中,\texttt{purchase\_book} 首先透過 \texttt{SELECT ... FOR UPDATE} 讀取並鎖定對應的 \texttt{posting} 紀錄與買家的 \texttt{"user"} 紀錄,確保同一時間只有一個交易流程可以修改同一筆商品與餘額資料,達成併行控制(Isolation)。接著,函數依序檢查:刊登是否存在、商品狀態是否為可購買的 \texttt{'listed'}、買家是否存在、餘額是否足夠,以及買家與賣家是否為同一人;只要其中任一條件不成立,函數便會立即回傳一個包含 \texttt{success = false} 與錯誤訊息的 JSON,不會對資料庫做任何修改。若所有檢查均通過,函數才會扣除買家餘額、增加賣家餘額、將對應的 \texttt{posting} 狀態更新為 \texttt{'sold'},並在 \texttt{orders} 表中建立一筆狀態為 \texttt{'completed'} 的訂單,最後回傳包含訂單編號與成交金額的成功結果。

為了讓金流紀錄與商品狀態自動同步,我們在 \texttt{004\_add\_triggers.sql} 中額外設計了多個與訂單相關的觸發器。例如,\texttt{update\_posting\_status\_on\_order} 會在 \texttt{orders} 表新增或更新且狀態為 \texttt{'completed'} 時,自動將對應的 \texttt{posting} 設為已售出;\texttt{record\_transaction\_on\_order} 則會在訂單完成時,同步於 \texttt{transaction\_record} 表中新增兩筆紀錄:一筆為買家的付款(負數金額,\texttt{trans\_type = 'payment'}),一筆為賣家的收入(正數金額,\texttt{trans\_type = 'income'}),形成類似雙向記帳的金流追蹤機制。上述函數與觸發器皆在同一資料庫交易中執行,當 \texttt{purchase\_book} 函數內部發生未預期錯誤時,PL/pgSQL 會進入 \texttt{EXCEPTION} 區塊並回傳失敗訊息,該次交易所做的更新將一併回滾(rollback),保證不會出現「扣了錢但沒有訂單」或「商品被標記為已售出但金流未入帳」等不一致情況。透過這樣的設計,我們將購買流程的關鍵不變性約束收斂在資料庫交易中,達成具備 ACID 特性的交易管理。

\subsection{並行控制}
在本系統中,使用者可能同時對同一筆刊登商品進行購買操作,因此若未妥善處理併行情況,
即可能發生「A 與 B 同時購買同一本書,結果兩人都扣款成功」或「帳款已異動但訂單未建立」
等資料不一致問題。為避免此類競態條件(Race Condition),我們將併行控制設計在資料庫層,
並以 PostgreSQL 所提供之行級鎖(Row-Level Lock)與交易(Transaction)機制進行保護。

所有購買流程皆由 \texttt{purchase\_book(p\_buyer\_id, p\_posting\_id)} 函數負責,
其程式碼可見於 \texttt{005\_add\_functions.sql}。當使用者提出購買請求時,系統會先對
目標刊登資料執行:

\texttt{SELECT ... FROM posting WHERE p\_id = p\_posting\_id FOR UPDATE;}

此語句會在資料庫層鎖定該筆刊登紀錄,使得若另一位使用者試圖同時購買同一筆商品,
後者操作將被阻塞,直到前一筆交易結束才可繼續。接著系統亦以相同方式鎖定買家餘額之
\texttt{"user"} 資料列,避免同一使用者在多個併行請求下產生「餘額重複扣除」的錯誤。

鎖定完成後,購買函數會在同一交易中依序執行餘額扣款、賣家入帳、更新貼文狀態並建立訂單。
若所有動作皆成功,交易即自動提交(Commit);若任一環節出現錯誤,PL/pgSQL 會進入
\texttt{EXCEPTION} 區塊並中止交易,所有變更將回滾(Rollback),確保不會產生部分更新成功、
部分失敗的狀態。透過此併行控制機制,系統可在高併發情境下維持資料一致性與交易正確性。

\section{分工資訊}
\begin{itemize}
    \item 資管三 B11705061 羅立宸:
    \item 資管三 B12705011 黃元翔:
    \item 資管三 B12705057 陳以倫:
\end{itemize}

\section{專案心得}
\subsection{資管三 B11705061 羅立宸}

\subsection{資管三 B12705011 黃元翔}

\subsection{資管三 B12705057 陳以倫}




\end{document}
